\documentclass[a4paper]{CSMakotoTechnicalReport}

\usepackage[spanish, es-noshorthands]{babel}
\usepackage{csquotes} 
\usepackage{amsmath} 
\usepackage{tikz}
\usetikzlibrary{arrows,automata,positioning}
\usepackage{tcolorbox}

\addbibresource{sample.bib} % BibLaTeX bibliography file

%----------------------------------------------------------------------------------------
%   REPORT INFORMATION
%----------------------------------------------------------------------------------------

\reporttitle{Procesador de MyJS: \textit{jsp}}

\reportsubtitle{Memoria del Grupo 59 (Primera Entrega)}

\reportauthors{Por Andrés Súnico (23M018)}

\reportdate{\today}

\leftheadercontent{Procesador de MyJS: \textit{jsp}}

%----------------------------------------------------------------------------------------

\begin{document}

%----------------------------------------------------------------------------------------
%   TITLE SECTION
%----------------------------------------------------------------------------------------

\thispagestyle{empty} % Suppress headers and footers on this page

\vspace*{0.08\textheight} % Vertical whitespace

{\large\raggedright\reportdate\par} % Report date

\vspace{0.01\textheight} % Vertical whitespace

{\fontsize{32pt}{34pt}\selectfont\raggedright\textbf{\reporttitle}\par} % Report title

\vspace{0.03\textheight} % Vertical whitespace

{\Large\raggedright\textit{\textbf{\reportsubtitle}}\par} % Subtitle

\vspace{0.06\textheight} % Vertical whitespace

{\large\raggedright\reportauthors\par} % Report authors, group or department

\vspace{0.36\textheight} % Vertical whitespace

%----------------------------------------------------------------------------------------

\begin{multicols}{2} % Begin two column mode

    %----------------------------------------------------------------------------------------
    %   TABLE OF CONTENTS
    %----------------------------------------------------------------------------------------

    \tableofcontents

    %----------------------------------------------------------------------------------------
    %   SECTIONS AND PARAGRAPHS
    %----------------------------------------------------------------------------------------

    \section{Introducción}

    El desarrollo del procesador \textit{jsp} se ha centrado en la experiencia del usuario (\textit{UX}), priorizando tres aspectos clave: una gestión de errores sólida y clara, una interfaz de línea de comandos (\textit{CLI}) intuitiva, y un rendimiento eficiente.

    Por ello, se ha elegido \textit{Rust} como el lenguaje de desarrollo. Ofrece una gestión de memoria eficiente, además de integrar \textit{clap}, una de las mejores librerías para desarollar aplicaciones \textit{CLI}.

    Gracias al uso del patrón de \textit{inyección de dependencias} en todo el proyecto, el código fuente es altamente extensible y modular.

    \section{Información Adicional}

    El código fuente del procesador se puede encontrar en \href{https://www.github.com/suuniqo}{github.com/suuniqo}, así como los tests y las dependencias del proyecto.

    \section{Opciones de la práctica}

    Además de las opciones comunes a todos los grupos, se han implementado las opciones:

    \subsection{Específicas del grupo}

    \begin{itemize}
        \item Comentarios de bloque (/* */)
        \item Cadenas con comillas dobles (" ")
        \item Sentencia repetitiva do-while
        \item Asignación con \textit{y} lógico (\textit{\&=})
        \item Análisis Sintáctico Ascendente
    \end{itemize}

    \subsection{Adicionales}

    Para que el procesador esté más completo, se han implementado adicionalmente todos los operadores lógicos, aritméticos, relacionales y unarios, así como los \textit{tokens} booleanos \textit{true} y \textit{false}.

    Además se ha escogido implementar el tratamiento de secuencias de escape (\textit{\textbackslash n} y \textit{\textbackslash t}).


    \section{Diseño del Lexer}
    
    El Analizador Léxico o \textit{Lexer} es uno de los 3 módulos principales del procesador.

    Al ser la primera capa de procesamiento, es el encargado de manejar el fichero fuente y convertirlo en una lista de \textit{tokens} para el Analizador Sintáctico.

    \subsection{Tokens}

    Con el fin de lograr un procesamiento eficiente, tanto en memoria como en complejidad, se han minimizado el número de \textit{tokens} con atributos.

    De este modo sólo 4 de un total de 41 \textit{tokens} van a utilizar un atributo.

    Cabe notar, además, que se ha decidido no hacer uso del \textit{token} fin de fichero (\textit{EOF}). Esto es porque el \textit{Lexer} se ha implementado como un iterador de \textit{tokens}, de modo que el final del flujo se detecta naturalmente cuando se consume el iterador.

    \renewcommand{\arraystretch}{1.19}
    \begin{table}[H]
        \caption{Listado de \textit{tokens}}
        \begin{tabular}{L{0.35\linewidth} L{0.31\linewidth} L{0.17\linewidth}}
            \toprule
            \textbf{Elemento} & \textbf{Código} & \textbf{Atributo} \\
            \midrule
            boolean & Bool & - \\
            do & Do & - \\
            float & Float & - \\
            function & Func & - \\
            if & If & - \\
            int & Int & - \\
            let & Let & - \\
            read & Read & - \\
            return & Ret & - \\
            string & Str & - \\
            void & Void & - \\
            while & While & - \\
            write & Write & - \\
            constante real & FloatLit & Número \\
            constante entera & IntLit & Número \\
            Cadena & StrLit & Cadena \\
            Identificador & Id & Posición \\
            \&= & AndAssign & - \\
            = & Assign & - \\
            , & Comma & - \\
            ; & Semi & - \\
            ( & LParen & - \\
            ) & RParen & - \\
            \{ & LBrack & - \\
            \} & RBrack & - \\
            Suma (+) & Sum & - \\
            Por (*) & Mul & - \\
            Resta (-) & Sub & - \\
            División (/) & Div & - \\
            Módulo (\%) & Mod & - \\
            Y lógico (\&\&) & And & - \\
            O lógico (||) & Or & - \\
            Negación (!) & Not & - \\
            Menor (<) & Lt & - \\
            Menor o igual (<=) & Le & - \\
            Mayor (>) & Gt & - \\
            Mayor o igual (>=) & Ge & - \\
            Relacionales: Distinto (!=) & Ne & - \\
            Igual (==) & Eq & - \\
            Menos Unario (-) & Sub & - \\
            Más Unario (+) & Sum & - \\
            false & False & - \\
            true & True & - \\
            \bottomrule
        \end{tabular}
        \label{tab:tokens}
    \end{table}

    \subsection{Errores}

    Cada tipo de error consta de un mensaje diferente y de una severidad, distinguiéndose \textit{error} de \textit{warning} (que no impediría la compilación del programa).

    El procesador genera mensajes claros con número de línea y columna, muestra la línea afectada y subraya en color la parte errónea.

    El \textit{Lexer} sólo genera un warning, \textit{Invalid Escape Sequence}. Como se muestra en \hyperref[subsec:acciones-semanticas]{Acciones Semánticas}, al detectar una secuencia de escape inválida no se descartara el \textit{token} cadena, sino que se conserva literalmente (por ejemplo, la secuencia \textit{\textbackslash q}, se sustituye por esos dos mismos caracteres).

    \renewcommand{\arraystretch}{1.35}
    \begin{table}[H]
        \caption{Listado de errores del \textit{Lexer}}
        \begin{tabular}{L{0.63\linewidth} L{0.20\linewidth}}
            \toprule
            \textbf{Error} & \textbf{Severidad} \\
            \midrule
            Carácter inválido & \textit{error} \\
            Comentario inacabado & \textit{error} \\
            Cadena inacabada & error \\
            Overflow de Cadena & \textit{error} \\
            Overflow de Entero & \textit{error} \\
            Overflow de Real & \textit{error} \\
            Formato de Real Inválido & \textit{error} \\
            Secuencia de Escape Inválida & \textit{warning} \\
            \bottomrule
        \end{tabular}
        \label{tab:err-lexer}
    \end{table}
\end{multicols}

    \subsection{Gramática}

    Se define la gramática del \textit{Lexer} como $G = (T, N, S, P)$, dónde:
    \begin{flalign*}
        T &= \{ \textnormal{Todo carácter \textit{ASCII}} \} \cup \{\textit{EOF}\} && \\
        N &= \{A, B, C, D, E, F, G, H, I, J, K, L, M\} &&
    \end{flalign*}
     $P$ se compone de las reglas:
    \begin{flalign*}
        S &\to delS \:\mid\: \textnormal{,} \:\mid\: \textnormal{;} \:\mid\: \textnormal{(} \:\mid\: \textnormal{)} \:\mid\: \textnormal{\{} \:\mid\: \textnormal{\}} \:\mid\: \textnormal{+} \:\mid\: \textnormal{*} \:\mid\: \textnormal{\%} \:\mid\: \textnormal{=}A \:\mid\: \textnormal{!}B \:\mid\: \textnormal{<}C \:\mid\: \textnormal{>}D \:\mid\: \textnormal{\&}E \:\mid\: \textnormal{|}F \:\mid\: dG \:\mid\: \textnormal{"}H \:\mid\: c_1I \:\mid\: \textnormal{/}J \:\mid\: \textit{EOF} && \\
        A &\to \: = \:\mid\: \lambda && \\
        B &\to \: = \:\mid\: \lambda && \\
        C &\to \: = \:\mid\: \lambda && \\
        D &\to \: = \:\mid\: \lambda && \\
        E &\to \: = \:\mid\: \lambda && \\
        F &\to \: | && \\
        G &\to dG \:\mid\: .K \:\mid\: \lambda && \\
        K &\to dK \:\mid\: \lambda && \\
        H &\to c_2H \:\mid\: \backslash L \:\mid\: " && \\
        L &\to \textnormal{n}H \:\mid\: \textnormal{t}H && \\
        I &\to c_3I \:\mid\: \lambda && \\
        J &\to *M \:\mid\: \lambda && \\
        M &\to c_4M \:\mid\: *N && \\
        N &\to c_5M \:\mid\: *N \:\mid\: /S &&
    \end{flalign*}
    Y se definen:
    \begin{flalign*}
        del &:= \{\textnormal{Todos los caracteres \textit{ASCII} \textit{whitespace}}\} && \\
        ngr &:= \{\textnormal{Todos los caracteres \textit{ASCII} no gráficos}\} && \\
        d &:= \{0, 1, \dots, 9\} && \\
        l &:= \{\textnormal{a}, \textnormal{b}, \dots, \textnormal{z}, \textnormal{A}, \textnormal{B}, \dots, \textnormal{Z}\} && \\
        c_1 &:= l \cup \{\_\} && \\
        c_2 &:= T \setminus (\{\backslash, ", \textit{EOF}\} \cup ngr) && \\
        c_3 &:= c_1 \cup d && \\
        c_4 &:= T \setminus \{*, \textit{EOF}\} && \\
        c_5 &:= T \setminus \{*, /, \textit{EOF}\} && \\
    \end{flalign*}


    \subsection{Autómata}

    A continuación se muestra el autómata finito determinista que reconoce el lenguaje generado por la gramática $G$. Nótese que una transición "o.c." ocurre al leer un carácter que no corresponda a otra transición del estado.

    Se considera un error y se detiene la ejecución cuando el autómata lee un carácter con el que no puede transitar. Solo se alcanza un estado final cuando se ha reconocido un \textit{token} exitosamente.

    Como se explica en el siguiente apartado, un autómata no va a ser un modelo suficientemente potente como para representar las operaciones de un \textit{Lexer}. Va a ser necesario complementarlo con algo más.

    \vspace{0.02\textheight}

    \begin{figure}[ht]
        \centering
        \begin{tikzpicture}[scale=0.15]
            \tikzstyle{every node}+=[inner sep=0pt]
            \draw [black] (42.4,-47.4) circle (3.7);
            \draw (42.4,-47.4) node {$0$};
            \draw [black] (21.2,-18.6) circle (3.7);
            \draw (21.2,-18.6) node {$1$};
            \draw [black] (21.2,-18.6) circle (3.1);
            \draw [black] (12.9,-21.4) circle (3.7);
            \draw (12.9,-21.4) node {$2$};
            \draw [black] (12.9,-21.4) circle (3.1);
            \draw [black] (8.4,-28.3) circle (3.7);
            \draw (8.4,-28.3) node {$3$};
            \draw [black] (8.4,-28.3) circle (3.1);
            \draw [black] (5.6,-36.9) circle (3.7);
            \draw (5.6,-36.9) node {$4$};
            \draw [black] (5.6,-36.9) circle (3.1);
            \draw [black] (3.9,-46) circle (3.7);
            \draw (3.9,-46) node {$5$};
            \draw [black] (3.9,-46) circle (3.1);
            \draw [black] (4.5,-56.3) circle (3.7);
            \draw (4.5,-56.3) node {$6$};
            \draw [black] (4.5,-56.3) circle (3.1);
            \draw [black] (6.8,-65.6) circle (3.7);
            \draw (6.8,-65.6) node {$7$};
            \draw [black] (6.8,-65.6) circle (3.1);
            \draw [black] (11.9,-73.9) circle (3.7);
            \draw (11.9,-73.9) node {$8$};
            \draw [black] (11.9,-73.9) circle (3.1);
            \draw [black] (21.2,-78.6) circle (3.7);
            \draw (21.2,-78.6) node {$9$};
            \draw [black] (21.2,-78.6) circle (3.1);
            \draw [black] (29.8,-78.6) circle (3.7);
            \draw (29.8,-78.6) node {$10$};
            \draw [black] (29.8,-78.6) circle (3.1);
            \draw [black] (50.4,-16.8) circle (3.7);
            \draw (50.4,-16.8) node {$11$};
            \draw [black] (62.6,-15.6) circle (3.7);
            \draw (62.6,-15.6) node {$12$};
            \draw [black] (72.8,-20.2) circle (3.7);
            \draw (72.8,-20.2) node {$13$};
            \draw [black] (85,-28.3) circle (3.7);
            \draw (85,-28.3) node {$14$};
            \draw [black] (92.1,-39.2) circle (3.7);
            \draw (92.1,-39.2) node {$15$};
            \draw [black] (66.6,-45.1) circle (3.7);
            \draw (66.6,-45.1) node {$16$};
            \draw [black] (74.3,-58.9) circle (3.7);
            \draw (74.3,-58.9) node {$17$};
            \draw [black] (81,-77.2) circle (3.7);
            \draw (81,-77.2) node {$18$};
            \draw [black] (62.6,-68.5) circle (3.7);
            \draw (62.6,-68.5) node {$19$};
            \draw [black] (40.3,-73.9) circle (3.7);
            \draw (40.3,-73.9) node {$20$};
            \draw [black] (40.3,-10.6) circle (3.7);
            \draw (40.3,-10.6) node {$21$};
            \draw [black] (40.3,-10.6) circle (3.1);
            \draw [black] (49.5,-4.8) circle (3.7);
            \draw (49.5,-4.8) node {$22$};
            \draw [black] (49.5,-4.8) circle (3.1);
            \draw [black] (59.8,-3.9) circle (3.7);
            \draw (59.8,-3.9) node {$23$};
            \draw [black] (59.8,-3.9) circle (3.1);
            \draw [black] (72,-8.6) circle (3.7);
            \draw (72,-8.6) node {$24$};
            \draw [black] (72,-8.6) circle (3.1);
            \draw [black] (81,-9.8) circle (3.7);
            \draw (81,-9.8) node {$25$};
            \draw [black] (81,-9.8) circle (3.1);
            \draw [black] (86.8,-17.7) circle (3.7);
            \draw (86.8,-17.7) node {$26$};
            \draw [black] (86.8,-17.7) circle (3.1);
            \draw [black] (95.3,-19.5) circle (3.7);
            \draw (95.3,-19.5) node {$27$};
            \draw [black] (95.3,-19.5) circle (3.1);
            \draw [black] (98.8,-28.9) circle (3.7);
            \draw (98.8,-28.9) node {$28$};
            \draw [black] (98.8,-28.9) circle (3.1);
            \draw [black] (104.2,-36.9) circle (3.7);
            \draw (104.2,-36.9) node {$29$};
            \draw [black] (104.2,-36.9) circle (3.1);
            \draw [black] (102.2,-46.6) circle (3.7);
            \draw (102.2,-46.6) node {$30$};
            \draw [black] (102.2,-46.6) circle (3.1);
            \draw [black] (82.7,-45.1) circle (3.7);
            \draw (82.7,-45.1) node {$31$};
            \draw [black] (82.7,-45.1) circle (3.1);
            \draw [black] (85,-54.2) circle (3.7);
            \draw (85,-54.2) node {$32$};
            \draw [black] (85,-65.6) circle (3.7);
            \draw (85,-65.6) node {$35$};
            \draw [black] (85,-65.6) circle (3.1);
            \draw [black] (96.2,-54.2) circle (3.7);
            \draw (96.2,-54.2) node {$33$};
            \draw [black] (96.2,-65.6) circle (3.7);
            \draw (96.2,-65.6) node {$34$};
            \draw [black] (96.2,-65.6) circle (3.1);
            \draw [black] (92.1,-77.2) circle (3.7);
            \draw (92.1,-77.2) node {$36$};
            \draw [black] (81,-87) circle (3.7);
            \draw (81,-87) node {$37$};
            \draw [black] (81,-87) circle (3.1);
            \draw [black] (62.6,-79.9) circle (3.7);
            \draw (62.6,-79.9) node {$38$};
            \draw [black] (62.6,-79.9) circle (3.1);
            \draw [black] (34.6,-64.6) circle (3.7);
            \draw (34.6,-64.6) node {$39$};
            \draw [black] (34.6,-64.6) circle (3.1);
            \draw [black] (40.3,-84.9) circle (3.7);
            \draw (40.3,-84.9) node {$40$};
            \draw [black] (52.6,-84.9) circle (3.7);
            \draw (52.6,-84.9) node {$41$};
            \draw [black] (41.806,-51.037) arc (18.45967:-269.54033:2.775);
            \draw (35.84,-54.67) node [below,align=center] {$del$};
            \fill [black] (39.15,-49.14) -- (38.23,-48.92) -- (38.55,-49.87);
            \draw [black] (43.6,-55.2) -- (42.96,-51.06);
            \fill [black] (42.96,-51.06) -- (42.59,-51.92) -- (43.58,-51.77);
            \draw [black] (24.534,-20.201) arc (61.28524:11.42915:34.675);
            \fill [black] (24.53,-20.2) -- (24.99,-21.02) -- (25.48,-20.15);
            \draw (36.38,-28.67) node [right,align=center] {$,$};
            \draw [black] (16.476,-22.345) arc (72.57177:24.64517:40.265);
            \fill [black] (16.48,-22.34) -- (17.09,-23.06) -- (17.39,-22.11);
            \draw (31.83,-30.06) node [above,align=center] {$;$};
            \draw [black] (12.079,-28.682) arc (81.70676:39.64176:44.871);
            \fill [black] (12.08,-28.68) -- (12.8,-29.29) -- (12.94,-28.3);
            \draw (28.41,-33.46) node [above,align=center] {$($};
            \draw [black] (9.192,-36.018) arc (100.80772:47.34262:35.397);
            \fill [black] (9.19,-36.02) -- (10.07,-36.36) -- (9.88,-35.38);
            \draw (26.21,-36.2) node [above,align=center] {$)$};
            \draw [black] (7.047,-44.058) arc (118.36416:57.47071:31.947);
            \fill [black] (7.05,-44.06) -- (7.99,-44.12) -- (7.51,-43.24);
            \draw (23.41,-39.71) node [above,align=center] {$\{$};
            \draw [black] (6.696,-53.326) arc (139.76528:66.6652:27.954);
            \fill [black] (6.7,-53.33) -- (7.59,-53.04) -- (6.83,-52.39);
            \draw (20.98,-43.59) node [above,align=center] {$\}$};
            \draw [black] (8.193,-62.175) arc (154.15624:79.99927:28.515);
            \fill [black] (8.19,-62.17) -- (8.99,-61.67) -- (8.09,-61.24);
            \draw (19.82,-48.71) node [above,align=center] {$+$};
            \draw [black] (12.317,-70.226) arc (169.72878:92.24287:27.928);
            \fill [black] (12.32,-70.23) -- (12.95,-69.53) -- (11.97,-69.35);
            \draw (20.64,-53.63) node [above,align=center] {$-$};
            \draw [black] (20.146,-75.057) arc (-168.05628:-260.33493:22.916);
            \fill [black] (20.15,-75.06) -- (20.47,-74.17) -- (19.49,-74.38);
            \draw (23.01,-56.09) node [left,align=center] {$*$};
            \draw [black] (27.819,-75.482) arc (-153.02437:-250.95789:19.453);
            \fill [black] (27.82,-75.48) -- (27.9,-74.54) -- (27.01,-75);
            \draw (26.37,-58.48) node [left,align=center] {$\%$};
            \draw [black] (42.125,-43.711) arc (-178.18269:-211.12003:43.44);
            \fill [black] (48.36,-19.88) -- (47.51,-20.31) -- (48.37,-20.83);
            \draw (42.75,-30.85) node [left,align=center] {$=$};
            \draw [black] (42.743,-43.717) arc (171.87908:123.27185:37.769);
            \fill [black] (59.41,-17.48) -- (58.47,-17.5) -- (59.02,-18.33);
            \draw (47.63,-27.5) node [left,align=center] {$!$};
            \draw [black] (43.559,-43.888) arc (158.92154:104.7188:37.734);
            \fill [black] (69.18,-20.96) -- (68.28,-20.68) -- (68.53,-21.65);
            \draw (52.53,-28.85) node [above,align=center] {$<$};
            \draw [black] (44.592,-44.421) arc (141.25458:87.04423:44.167);
            \fill [black] (81.32,-27.95) -- (80.54,-27.41) -- (80.49,-28.41);
            \draw (59.93,-31.25) node [above,align=center] {$>$};
            \draw [black] (45.476,-45.345) arc (121.87872:76.85893:56.985);
            \fill [black] (88.53,-38.24) -- (87.86,-37.57) -- (87.63,-38.55);
            \draw (65.74,-36.89) node [above,align=center] {$\&$};
            \draw [black] (45.985,-46.485) arc (102.73111:88.12721:66.873);
            \fill [black] (62.91,-44.88) -- (62.12,-44.35) -- (62.09,-45.35);
            \draw (54.27,-44.59) node [above,align=center] {$|$};
            \draw [black] (46.075,-47.829) arc (81.74377:58.6076:66.585);
            \fill [black] (71.2,-56.89) -- (70.77,-56.04) -- (70.25,-56.9);
            \draw (60.02,-50.56) node [above,align=center] {$d$};
            \draw [black] (45.818,-48.815) arc (66.278:38.38423:86.347);
            \fill [black] (78.77,-74.25) -- (78.66,-73.31) -- (77.88,-73.93);
            \draw (64.76,-59.02) node [above,align=center] {$"$};
            \draw [black] (45.419,-49.539) arc (53.10827:34.39498:67.492);
            \fill [black] (60.59,-65.39) -- (60.56,-64.45) -- (59.73,-65.01);
            \draw (54.18,-55.37) node [right,align=center] {$c_1$};
            \draw [black] (44.424,-50.491) arc (27.75739:-36.81929:19.414);
            \fill [black] (42.79,-71.17) -- (43.67,-70.83) -- (42.86,-70.23);
            \draw (47.21,-61.12) node [right,align=center] {$/$};
            \draw [black] (47.25,-14.86) -- (43.45,-12.54);
            \fill [black] (43.45,-12.54) -- (43.87,-13.38) -- (44.4,-12.53);
            \draw (46.41,-13.2) node [above,align=center] {$=$};
            \draw [black] (50.12,-13.11) -- (49.78,-8.49);
            \fill [black] (49.78,-8.49) -- (49.34,-9.32) -- (50.34,-9.25);
            \draw (50.56,-10.75) node [right,align=center] {$o.c.$};
            \draw [black] (61.74,-12) -- (60.66,-7.5);
            \fill [black] (60.66,-7.5) -- (60.36,-8.39) -- (61.33,-8.16);
            \draw (61.96,-9.32) node [right,align=center] {$=$};
            \draw [black] (65.57,-13.39) -- (69.03,-10.81);
            \fill [black] (69.03,-10.81) -- (68.09,-10.89) -- (68.69,-11.69);
            \draw (65.6,-11.6) node [above,align=center] {$o.c$};
            \draw [black] (75.09,-17.29) -- (78.71,-12.71);
            \fill [black] (78.71,-12.71) -- (77.82,-13.02) -- (78.61,-13.64);
            \draw (77.47,-16.42) node [right,align=center] {$=$};
            \draw [black] (76.44,-19.55) -- (83.16,-18.35);
            \fill [black] (83.16,-18.35) -- (82.28,-18) -- (82.46,-18.98);
            \draw (80.62,-19.61) node [below,align=center] {$o.c$};
            \draw [black] (87.81,-25.9) -- (92.49,-21.9);
            \fill [black] (92.49,-21.9) -- (91.55,-22.04) -- (92.2,-22.8);
            \draw (91.22,-24.39) node [below,align=center] {$=$};
            \draw [black] (88.7,-28.46) -- (95.1,-28.74);
            \fill [black] (95.1,-28.74) -- (94.33,-28.21) -- (94.28,-29.2);
            \draw (91.83,-29.17) node [below,align=center] {$o.c$};
            \draw [black] (95.73,-38.51) -- (100.57,-37.59);
            \fill [black] (100.57,-37.59) -- (99.69,-37.25) -- (99.87,-38.23);
            \draw (98.71,-38.65) node [below,align=center] {$=$};
            \draw [black] (95.08,-41.39) -- (99.22,-44.41);
            \fill [black] (99.22,-44.41) -- (98.87,-43.54) -- (98.27,-44.34);
            \draw (95.87,-43.4) node [below,align=center] {$\&$};
            \draw [black] (70.3,-45.1) -- (79,-45.1);
            \fill [black] (79,-45.1) -- (78.2,-44.6) -- (78.2,-45.6);
            \draw (74.65,-45.6) node [below,align=center] {$|$};
            \draw [black] (72.048,-55.982) arc (245.39558:-42.60442:2.775);
            \draw (72.09,-50.15) node [above,align=center] {$d$};
            \fill [black] (75.25,-55.34) -- (76.03,-54.82) -- (75.12,-54.4);
            \draw [black] (77.69,-57.41) -- (81.61,-55.69);
            \fill [black] (81.61,-55.69) -- (80.68,-55.55) -- (81.08,-56.47);
            \draw (78.92,-56.04) node [above,align=center] {$.$};
            \draw [black] (77.44,-60.86) -- (81.86,-63.64);
            \fill [black] (81.86,-63.64) -- (81.45,-62.79) -- (80.92,-63.64);
            \draw (77.71,-62.75) node [below,align=center] {$o.c.$};
            \draw [black] (88.7,-54.2) -- (92.5,-54.2);
            \fill [black] (92.5,-54.2) -- (91.7,-53.7) -- (91.7,-54.7);
            \draw (90.6,-54.7) node [below,align=center] {$d$};
            \draw [black] (99.852,-53.707) arc (125.41506:-162.58494:2.775);
            \draw (104.86,-57.56) node [right,align=center] {$d$};
            \fill [black] (98.81,-56.8) -- (98.87,-57.74) -- (99.68,-57.16);
            \draw [black] (96.2,-57.9) -- (96.2,-61.9);
            \fill [black] (96.2,-61.9) -- (96.7,-61.1) -- (95.7,-61.1);
            \draw (95.7,-59.9) node [left,align=center] {$o.c.$};
            \draw [black] (78.755,-80.123) arc (-9.80292:-297.80292:2.775);
            \draw (72.83,-81.97) node [left,align=center] {$c_2$};
            \fill [black] (77.31,-77.2) -- (76.61,-76.57) -- (76.44,-77.55);
            \draw [black] (83.876,-74.943) arc (112.76955:67.23045:6.908);
            \fill [black] (89.22,-74.94) -- (88.68,-74.17) -- (88.29,-75.09);
            \draw (86.55,-73.91) node [above,align=center] {$\backslash$};
            \draw [black] (88.872,-78.947) arc (-74.09162:-105.90838:8.471);
            \fill [black] (84.23,-78.95) -- (84.86,-79.65) -- (85.13,-78.69);
            \draw (86.55,-79.77) node [below,align=center] {$n,t$};
            \draw [black] (81,-80.9) -- (81,-83.3);
            \fill [black] (81,-83.3) -- (81.5,-82.5) -- (80.5,-82.5);
            \draw (80.5,-82.1) node [left,align=center] {$"$};
            \draw [black] (66.268,-68.861) arc (112.11578:-175.88422:2.775);
            \draw (70.3,-74.19) node [right,align=center] {$c_3$};
            \fill [black] (64.54,-71.63) -- (64.38,-72.56) -- (65.31,-72.18);
            \draw [black] (62.6,-72.2) -- (62.6,-76.2);
            \fill [black] (62.6,-76.2) -- (63.1,-75.4) -- (62.1,-75.4);
            \draw (62.1,-74.2) node [left,align=center] {$o.c.$};
            \draw [black] (38.37,-70.75) -- (36.53,-67.75);
            \fill [black] (36.53,-67.75) -- (36.53,-68.7) -- (37.38,-68.18);
            \draw (38.09,-67.97) node [right,align=center] {$o.c.$};
            \draw [black] (40.3,-77.6) -- (40.3,-81.2);
            \fill [black] (40.3,-81.2) -- (40.8,-80.4) -- (39.8,-80.4);
            \draw (39.8,-79.4) node [left,align=center] {$*$};
            \draw [black] (38.921,-88.318) arc (5.7603:-282.2397:2.775);
            \draw (33.55,-91.9) node [left,align=center] {$c_4$};
            \fill [black] (36.75,-85.88) -- (35.9,-85.47) -- (36,-86.46);
            \draw [black] (43.255,-82.727) arc (113.24615:66.75385:8.096);
            \fill [black] (49.65,-82.73) -- (49.11,-81.95) -- (48.71,-82.87);
            \draw (46.45,-81.57) node [above,align=center] {$*$};
            \draw [black] (49.793,-87.254) arc (-63.93898:-116.06102:7.61);
            \fill [black] (43.11,-87.25) -- (43.61,-88.06) -- (44.05,-87.16);
            \draw (46.45,-88.53) node [below,align=center] {$c_5$};
            \draw [black] (56.207,-85.657) arc (105.87903:-182.12097:2.775);
            \draw (59.72,-91.45) node [right,align=center] {$*$};
            \fill [black] (54.19,-88.22) -- (53.93,-89.13) -- (54.89,-88.86);
            \draw [black] (45.122,-49.904) arc (44.2371:-13.80445:33.602);
            \fill [black] (45.12,-49.9) -- (45.32,-50.83) -- (46.04,-50.13);
            \draw (54.24,-64) node [right,align=center] {$/$};
        \end{tikzpicture}
        \vspace{0.03\textheight}
        \caption{Autómata que reconoce el lenguaje $L(G)$}
        \label{fig:automaton}
    \end{figure}

    \vspace{0.01\textheight}

    \subsection{Acciones Semánticas}
    \label{subsec:acciones-semanticas}

    Las acciones semánticas son operaciones adicionales que se ejecutan durante las transiciones del autómata, con el propósito de aumentar la expresividad cuando es necesario. Resultan especialmente útiles para realizar conversiones de tipos o para simplificar la ejecución de otras acciones más complejas.

    Por claridad, se dividen en varios grupos:

    \subsubsection{General}

    \lstset{
        aboveskip=8pt,
        belowskip=8pt,
        basicstyle=\ttfamily\small,
        breaklines=true,
        frame=single
    }

    \begin{description}
        \item[READ] Segunda acción de toda transición menos $11$:$22$, $12$:$24$, $13$:$26$, $14$:$28$, $17$:$35$, $33$:$34$, $36$:$18$, $19$:$38$, $20$:$39$
            \begin{lstlisting}[language=C]
 chr := read()
            \end{lstlisting}

        \item[INV\_CHAR] Ante cualquier error no manejado en el resto de acciones
            \begin{lstlisting}[language=C]
 error("Illegal character")
            \end{lstlisting}
    \end{description}

    \subsubsection{Generación Directa}

    \begin{multicols}{2}

    \begin{description}
        \item[GEN\_COMMA] En la transición $0$:$1$
            \begin{lstlisting}[language=C]
 gen_token(Comma, -)
            \end{lstlisting}

        \item[GEN\_SEMI] En la transición $0$:$2$
            \begin{lstlisting}[language=C]
 gen_token(Semi, -)
            \end{lstlisting}

        \item[GEN\_LPAREN] En la transición $0$:$3$
            \begin{lstlisting}[language=C]
 gen_token(LParen, -)
            \end{lstlisting}

        \item[GEN\_RPAREN] En la transición $0$:$4$
            \begin{lstlisting}[language=C]
 gen_token(RParen, -)
            \end{lstlisting}

        \item[GEN\_LBRACK] En la transición $0$:$5$
            \begin{lstlisting}[language=C]
 gen_token(LBrack, -)
            \end{lstlisting}

        \item[GEN\_RBRACK] En la transición $0$:$6$
            \begin{lstlisting}[language=C]
 gen_token(RBrack, -)
            \end{lstlisting}

        \item[GEN\_SUM] En la transición $0$:$7$
            \begin{lstlisting}[language=C]
 gen_token(Sum, -)
            \end{lstlisting}

        \item[GEN\_SUB] En la transición $0$:$8$
            \begin{lstlisting}[language=C]
 gen_token(Sub, -)
            \end{lstlisting}

        \item[GEN\_MUL] En la transición $0$:$9$
            \begin{lstlisting}[language=C]
 gen_token(Mul, -)
            \end{lstlisting}

        \item[GEN\_MOD] En la transición $0$:$10$
            \begin{lstlisting}[language=C]
 gen_token(Mod, -)
            \end{lstlisting}

        \item[GEN\_EQ] En la transición $11$:$21$
            \begin{lstlisting}[language=C]
 gen_token(Eq, -)
            \end{lstlisting}

        \item[GEN\_ASSIGN] En la transición $11$:$22$
            \begin{lstlisting}[language=C]
 gen_token(Assign, -)
            \end{lstlisting}

        \item[GEN\_NE] En la transición $12$:$23$
            \begin{lstlisting}[language=C]
 gen_token(Ne, -)
            \end{lstlisting}

        \item[GEN\_NOT] En la transición $12$:$24$
            \begin{lstlisting}[language=C]
 gen_token(Not, -)
            \end{lstlisting}

        \item[GEN\_LE] En la transición $13$:$25$
            \begin{lstlisting}[language=C]
 gen_token(Le, -)
            \end{lstlisting}

        \item[GEN\_LT] En la transición $13$:$26$
            \begin{lstlisting}[language=C]
 gen_token(Lt, -)
            \end{lstlisting}

        \item[GEN\_GE] En la transición $14$:$27$
            \begin{lstlisting}[language=C]
 gen_token(Ge, -)
            \end{lstlisting}

        \item[GEN\_GT] En la transición $14$:$28$
            \begin{lstlisting}[language=C]
 gen_token(Gt, -)
            \end{lstlisting}

        \item[GEN\_ANDASSIGN] En la transición $15$:$29$
            \begin{lstlisting}[language=C]
 gen_token(AndAssign, -)
            \end{lstlisting}

        \item[GEN\_AND] En la transición $15$:$30$
            \begin{lstlisting}[language=C]
 gen_token(And, -)
            \end{lstlisting}

        \item[GEN\_OR] En la transición $16$:$31$
            \begin{lstlisting}[language=C]
 gen_token(Or, -)
            \end{lstlisting}

        \item[GEN\_DIV] En la transición $20$:$39$
            \begin{lstlisting}[language=C]
 gen_token(Div, -)
            \end{lstlisting}

    \end{description}
    \end{multicols}

    \begin{minipage}{\textwidth}
    \subsubsection{Generación de Números}

    \begin{multicols}{2}

    \begin{description}
        \item[INIT\_NUM] En la transición $0$:$17$
            \begin{lstlisting}[language=C]
 num := val(chr)
            \end{lstlisting}

        \item[INIT\_DEC] En la transición $32$:$33$
            \begin{lstlisting}[language=C]
 if (!is_ascii_digit(chr)) {
     error("Invalid Float Format")
 } else {
     dec := 10
     num := num + vald(chr) / dec
 }
            \end{lstlisting}

        \item[GEN\_DEC] En la transición $33$:$34$
            \begin{lstlisting}[language=C]
 if (size_bytes(num) > 16) {
     error("Float out of range")
 } else {
     gen_token(FloatLit, num)
 }
            \end{lstlisting}

            \columnbreak

        \item[ADD\_INTDIG] En la transición $17$:$17$
            \begin{lstlisting}[language=C]
 num := num * 10 + val(chr)
            \end{lstlisting}

            \vspace{0.03\textheight}
        \item[ADD\_DECDIG] En las transiciones $33$:$33$
            \begin{lstlisting}[language=C]
 dec := dec * 10
 num := num + vald(chr) / dec
            \end{lstlisting}

            \vspace{0.02\textheight}
        \item[GEN\_INT] En la transición $17$:$35$
            \begin{lstlisting}[language=C]
 if (size_bytes(num) > 16) {
     error("Integer out of range")
 } else {
     gen_token(IntLit, num)
 }
            \end{lstlisting}

    \end{description}
    \end{multicols}

    \subsubsection{Generación de Cadenas e Identificadores}

    \begin{multicols}{2}
    \begin{description}

        \item[INIT\_STR\_ID] En la transición $0$:$18$, $0$:$19$
            \begin{lstlisting}[language=C]
 lex := ""
            \end{lstlisting}

        \item[ADD\_CHAR\_STR] En la transición $18$:$18$
            \begin{lstlisting}[language=C]
 if (chr == EOF) {
     error("Unterminated String")
 } else {
     lex.concat(chr)
 }
            \end{lstlisting}

        \item[ADD\_ESCSEQ] En la transición $36$:$18$
            \begin{lstlisting}[language=C]
 switch (chr) {
     case 'n' -> lex.concat('\n')
     case 't' -> lex.concat('\t')
     case EOF -> {
         error("Unfinsihed comment")
     }
     default  -> {
         warning("Invalid sequence")
         lex.concat('\\')
         lex.concat(chr)
     }
 }
            \end{lstlisting}

        \columnbreak

        \item[ADD\_CHAR\_ID] En la transición $0$:$19$, $19$:$19$
            \begin{lstlisting}[language=C]
 lex.concat(chr)
            \end{lstlisting}
        \item[GEN\_STR] En la transición $18$:$37$
            \begin{lstlisting}[language=C]
 if (lex.len() > 64) {
     error("String is too long")
 } else {
     gen_token(StrLit, lex)
 }
            \end{lstlisting}

        \item[GEN\_ID] En la transición $19$:$38$
            \begin{lstlisting}[language=C]
 code := search_keyword(lex)

 if (code != null) {
     gen_token(code, -)
 } else {
     pos := symtable_search(lex)

     if (pos == null) {
         pos := symtable_insert(lex)
     }
     gen_token(Id, pos)
 }
            \end{lstlisting}

    \end{description}
    \end{multicols}

    \subsubsection{Procesado de Comentarios}

    \begin{description}
        \item[UNTERM\_COMM] En las transiciones $40$:$40$, $40$:$41$, $41$:$40$, $41$:$41$
            \begin{lstlisting}[language=C]
 if (chr == EOF) {
     error("Unterminated Comment")
 }
            \end{lstlisting}
    \end{description}
    \end{minipage}

    \section{Diseño de la Tabla de Símbolos}

    Se trata de un tipo abstracto de datos encargado de gestionar la información relevante a los identificadores del programa. Todos los módulos del procesador van a necesitar acceder a ella con distintos propósitos por lo que es importante que tanto la inserción como la consulta de datos sea eficiente.

    \subsection{Estructura y Organización}

    \subsubsection{Entradas}

    La información de los identificadores se va a guardar en la tabla de símbolos en forma de entradas. Como en \textit{MyJS} no existen los \textit{arrays} se distinguen únicamente 2 tipos:

    \begin{description}
        \item[Entrada Básica] Para todos los tipos básicos, es decir, \textit{int}, \textit{float}, \textit{string} y \textit{bool}.
            \begin{description}
                \item[Lexema] Nombre de la variable.
                \item[Tipo] Tipo de la variable.
                \item[Desplazamiento] Desplazamiento en memoria relativo a su ámbito.
            \end{description}

        \item[Entrada Función] Para las funciones. Nótese que 'Tipos Argumentos' es un puntero a una lista de tipos.
            \begin{description}
                \item[Lexema] Nombre de la función.
                \item[Tipo Retorno] Tipo que devuelve la función, pudiendo ser \textit{Void}.
                \item[Tipos Argumentos] Lista de los tipos de los parámetros en orden.
                \item[Etiqueta] Etiqueta que se usará para navegar a la función en el código ensamblador.
            \end{description}
    \end{description}

    Cada entrada va a tener una estructura de 'llave-valor', dónde el lexema del identificador actúa como llave, y sus atributos (toda su información relevante) como valor. Como cada llave identifica de forma única cada entrada, se puede optimizar el complejidad de acceso e inserción a $O(1)$ usando \textit{hashmaps}.


    \subsubsection{Ámbitos}

    No siempre se puede acceder a cada variable de un programa. Por ejemplo, desde una función no se puede acceder a una variable local de otra. Por ello, por cada ámbito se va a crear una tabla de símbolos distinta. Además, como \textit{MyJS} es un lenguaje sin anidamiento de funciones, en cada momento habrá como máximo 2 tablas de símbolos activas: la global y, opcionalmente, la de una función.
    
    De esta manera, se puede comprender una tabla de símbolos como una \textit{stack} de ámbitos (es decir, tablas de símbolos locales), dónde el Analizador Semántico será el encargado de apilar y desapilar ámbitos al entrar y salir de funciones respectivamente.

    \section{Diseño del Parser}
    Se detallará en la próxima entrega.

    \section{Diseño del Semanter}
    Se detallará en la entrega final.

    \section{Diseño del Gestor de Errores}
    Se detallará en la entrega final.

%----------------------------------------------------------------------------------------
%   APPENDICES
%----------------------------------------------------------------------------------------

\newpage

\section*{Anexo}

\begin{appendices}

    \section{Casos de Prueba}

    Se va a probar el funcionamiento del procesador con 6 fichero fuentes distintos. La mitad de ellos serán correctos y la otra incorrectos. En los casos correctos se volcará el fichero de \textit{tokens} y de la tabla de símbolos; en los incorrectos los diagnósticos generados por el gestor de errores.

    \subsection{Casos Correctos}

    \begin{description}
        \item[fib.javascript] Fichero con un estilo limpio y estándar.
            \begin{lstlisting}[language=C]
/* This function computes the nth fibonacci number */
function int fib(int n) {
    if (n == 0) {
        return 0;
    }
    if (n == 1) {
        return 1;
    }

    let int a = 0;
    let int b = 1;

    do {
        let int c = a + b;

        a = b;
        b = c;

        n = n - 1;
    } while (n >= 2);

    return b;
}
            \end{lstlisting}
            \begin{multicols}{4}
                \textbf{Fichero de \textit{tokens}:}
                \begin{lstlisting}
<Func, >
<Int, >
<Id, 15>
<LParen, >
<Int, >
<Id, 16>
<RParen, >
<LBrack, >
<If, >
<LParen, >
<Id, 16>
<Eq, >
<IntLit, 0>
<RParen, >
<LBrack, >
<Ret, >
<IntLit, 0>
<Semi, >
<RBrack, >
<If, >
<LParen, >
<Id, 16>
<Eq, >
<IntLit, 1>
<RParen, >
<LBrack, >
<Ret, >
<IntLit, 1>
<Semi, >
<RBrack, >
<Let, >
<Int, >
<Id, 17>
<Assign, >
<IntLit, 0>
<Semi, >
<Let, >
<Int, >
<Id, 18>
<Assign, >
<IntLit, 1>
<Semi, >
<Do, >
<LBrack, >
<Let, >
<Int, >
<Id, 19>
<Assign, >
<Id, 17>
<Sum, >
<Id, 18>
<Semi, >
<Id, 17>
<Assign, >
<Id, 18>
<Semi, >
<Id, 18>
<Assign, >
<Id, 19>
<Semi, >
<Id, 16>
<Assign, >
<Id, 16>
<Sub, >
<IntLit, 1>
<Semi, >
<RBrack, >
<While, >
<LParen, >
<Id, 16>
<Ge, >
<IntLit, 2>
<RParen, >
<Semi, >
<Ret, >
<Id, 18>
<Semi, >
<RBrack, >
                \end{lstlisting}
                \textbf{Tabla de símbolos:}
                \begin{lstlisting}
table #0:
* 'fib'
* 'n'
* 'a'
* 'b'
* 'c'
                \end{lstlisting}
            \end{multicols}

        \item[factorial.javascript] Fichero con un código más comprimido y comentarios más raros.
            \begin{lstlisting}[language=C]
/*********************************************/
/*          * / * N FACTORIAL * / *          */
/*********************************************/

/*
* This function computes n! (n factorial)
*/
function int factorial(int n) {
    if(n==0){return 1;}
    if(n<=2){return n;}
    let int res=n;
    do{n=n-1;res=res*n;}while(n>2);
    return res;
}

/* read n and write n! */
let int n=0;read(n);let int res=factorial(n);write("n! = ", res);

/*** eof ***/
            \end{lstlisting}
            \begin{multicols}{3}
                \textbf{Fichero de \textit{tokens}:}
                \begin{lstlisting}
<Func, >
<Int, >
<Id, 15>
<LParen, >
<Int, >
<Id, 16>
<RParen, >
<LBrack, >
<If, >
<LParen, >
<Id, 16>
<Eq, >
<IntLit, 0>
<RParen, >
<LBrack, >
<Ret, >
<IntLit, 1>
<Semi, >
<RBrack, >
<If, >
<LParen, >
<Id, 16>
<Le, >
<IntLit, 2>
<RParen, >
<LBrack, >
<Ret, >
<Id, 16>
<Semi, >
<RBrack, >
<Let, >
<Int, >
<Id, 17>
<Assign, >
<Id, 16>
<Semi, >
<Do, >
<LBrack, >
<Id, 16>
<Assign, >
<Id, 16>
<Sub, >
<IntLit, 1>
<Semi, >
<Id, 17>
<Assign, >
<Id, 17>
<Mul, >
<Id, 16>
<Semi, >
<RBrack, >
<While, >
<LParen, >
<Id, 16>
<Gt, >
<IntLit, 2>
<RParen, >
<Semi, >
<Ret, >
<Id, 17>
<Semi, >
<RBrack, >
<Let, >
<Int, >
<Id, 16>
<Assign, >
<IntLit, 0>
<Semi, >
<Read, >
<LParen, >
<Id, 16>
<RParen, >
<Semi, >
<Let, >
<Int, >
<Id, 17>
<Assign, >
<Id, 15>
<LParen, >
<Id, 16>
<RParen, >
<Semi, >
<Write, >
<LParen, >
<StrLit, "n! = ">
<Comma, >
<Id, 17>
<RParen, >
<Semi, >
                \end{lstlisting}
                \textbf{Tabla de símbolos:}
                \begin{lstlisting}
table #0:
* 'factorial'
* 'n'
* 'res'
                \end{lstlisting}
            \end{multicols}

        \newpage

        \item[fuzz.javascript] Fichero correcto pero caótico, con el objectivo de obtener \textit{edgecases}.
            \begin{lstlisting}[language=C]
let int global=13;
function void nothing(void) { read(global) do{global=global-1;}while(global>0);}
function int main(float b, string d){let string aux="this is a string";
/*** this is a comment ***/ let float foo=1203.123; let int bar=2398;
;;;/* semicolons */;;; let oper=2||5;let boolean bool=false;
let boolean george=true;george&=bool;}/*** / eof /***/
            \end{lstlisting}
            \begin{multicols}{3}
                \textbf{Fichero de \textit{tokens}:}
                \begin{lstlisting}
<Let, >
<Int, >
<Id, 15>
<Assign, >
<IntLit, 13>
<Semi, >
<Func, >
<Void, >
<Id, 16>
<LParen, >
<Void, >
<RParen, >
<LBrack, >
<Read, >
<LParen, >
<Id, 15>
<RParen, >
<Do, >
<LBrack, >
<Id, 15>
<Assign, >
<Id, 15>
<Sub, >
<IntLit, 1>
<Semi, >
<RBrack, >
<While, >
<LParen, >
<Id, 15>
<Gt, >
<IntLit, 0>
<RParen, >
<Semi, >
<RBrack, >
<Func, >
<Int, >
<Id, 17>
<LParen, >
<Float, >
<Id, 18>
<Comma, >
<Str, >
<Id, 19>
<RParen, >
<LBrack, >
<Let, >
<Str, >
<Id, 20>
<Assign, >
<StrLit, "this is a string">
<Semi, >
<Let, >
<Float, >
<Id, 21>
<Assign, >
<FloatLit, 1203.123>
<Semi, >
<Let, >
<Int, >
<Id, 22>
<Assign, >
<IntLit, 2398>
<Semi, >
<Semi, >
<Semi, >
<Semi, >
<Semi, >
<Semi, >
<Semi, >
<Let, >
<Id, 23>
<Assign, >
<IntLit, 2>
<Or, >
<IntLit, 5>
<Semi, >
<Let, >
<Bool, >
<Id, 24>
<Assign, >
<False, >
<Semi, >
<Let, >
<Bool, >
<Id, 25>
<Assign, >
<True, >
<Semi, >
<Id, 25>
<AndAssign, >
<Id, 24>
<Semi, >
<RBrack, >
                \end{lstlisting}
                \textbf{Tabla de símbolos:}
                \begin{lstlisting}
table #0:
* 'global'
* 'nothing'
* 'main'
* 'b'
* 'd'
* 'aux'
* 'foo'
* 'bar'
* 'oper'
* 'bool'
* 'george'
                \end{lstlisting}
            \end{multicols}
    \end{description}

    \newpage

    \subsection{Casos Incorrectos}
    \begin{description}
        \item[unterm.javascript] Fichero con errores de sentencias incompletas.
            \begin{lstlisting}
/* unfinished strings */
let string foo = "im not finishing this string
let string bar = "im not finishing this one either\
let string rep = "last one\q

/* unfinished floats */
let float x = 478234.
let float y = 78234..
let float z = 2343...

/*** * / * ///** / this comment is finished /* /* ***/
/*** this one is not ** ****** *** /*

function int main {
    let string useless = "this variable won't be recognised";
    return 0;
}
            \end{lstlisting}

            \textbf{Diagnósticos:}
            \begin{lstlisting}
unterm.javascript:2:18: error: missing terminating character '"' on string literal
   2 | let string foo = "im not finishing this string
     |                  ^~~~~~~~~~~~~~~~~~~~~~~~~~~~~
unterm.javascript:3:18: error: missing terminating character '"' on string literal
   3 | let string bar = "im not finishing this one either\
     |                  ^~~~~~~~~~~~~~~~~~~~~~~~~~~~~~~~~~
unterm.javascript:4:27: warning: unknown escape sequence '\q'
   4 | let string rep = "last one\q
     |                           ^~
unterm.javascript:4:18: error: missing terminating character '"' on string literal
   4 | let string rep = "last one\q
     |                  ^~~~~~~~~~~
unterm.javascript:7:15: error: expected digit after '.' in float literal
   7 | let float x = 478234.
     |               ^~~~~~~
unterm.javascript:8:15: error: expected digit after '.' in float literal
   8 | let float y = 78234..
     |               ^~~~~~
unterm.javascript:8:21: error: illegal character '.' in program
   8 | let float y = 78234..
     |                     ^
unterm.javascript:9:15: error: expected digit after '.' in float literal
   9 | let float z = 2343...
     |               ^~~~~
unterm.javascript:9:20: error: illegal character '.' in program
   9 | let float z = 2343...
     |                    ^
unterm.javascript:9:21: error: illegal character '.' in program
   9 | let float z = 2343...
     |                     ^
unterm.javascript:12:1: error: unterminated comment
   12 | /*** this one is not ** ****** *** /*
      | ^~
            \end{lstlisting}
            \newpage
        
        \item[overflow.javascript] Fichero con errores de constantes fuera del rango admitido.
            \begin{lstlisting}
/* string with 64 characters */
let string foo = "ffffffffffffffffffffffffffffffffffffffffffffffffffffffffffffffff";

/* string with 65 characters */
let string bar = "bbbbbbbbbbbbbbbbbbbbbbbbbbbbbbbbbbbbbbbbbbbbbbbbbbbbbbbbbbbbbbbbb";

/* max int16 */
let int imax = 32767;
let int imax_plus_one = 32768;

/* max float */
let float fmax = 340282346638528859811704183484516925440.0;
let float fmax_plus_one = 340282346638528859811704183484516925441;
let float lots_of_decimals = 0.2347028349820934809218409238845290380928;
            \end{lstlisting}

            \textbf{Diagnósticos:}
            \begin{lstlisting}
overflow.javascript:5:18: error: string literal is too long, length is 65 but the maximum is 64
   5 | let string bar = "bbbbbbbbbbbbbbbbbbbbbbbbbbbbbbbbbbbbbbbbbbbbbbbbbbbbbbbbbbbbbbbbb";
     |                  ^~~~~~~~~~~~~~~~~~~~~~~~~~~~~~~~~~~~~~~~~~~~~~~~~~~~~~~~~~~~~~~~~~~
overflow.javascript:9:25: error: integer literal out of range for 16-byte type
   9 | let int imax_plus_one = 32768;
     |                         ^~~~~
overflow.javascript:13:27: error: integer literal out of range for 16-byte type
  13 | let float fmax_plus_one = 340282346638528859811704183484516925441;
     |                           ^~~~~~~~~~~~~~~~~~~~~~~~~~~~~~~~~~~~~~~
            \end{lstlisting}
            \newpage

        \item[noise.javascript] Fichero de ruido, repleto de errores diferentes.
            \begin{lstlisting}
123.45.67 12abc 12.34.56..78
-+*/% &&& ||| != == <= >= >< =!= &== |= <=>
"unterminated string" "string with illegal char \q" ""empty""
"bad\escape" "no closing quote
/* nested comment */ /* comment */ /* unclosed comment
@@@ ### $$$ %%% ^^^ ??? !!! <<< >>> === ====
++ -- ** // %% =!= &== |= != <=> &&||!!
"unterminated string again /* unclosed comment
"bad string with \x illegal escape" "" ""empty""
/* final comment */ /* another unclosed
            \end{lstlisting}

            \textbf{Diagnósticos:}
            \begin{lstlisting}
noise.javascript:1:7: error: illegal character '.' in program
   1 | 123.45.67 12abc 12.34.56..78
     |       ^
noise.javascript:1:22: error: illegal character '.' in program
   1 | 123.45.67 12abc 12.34.56..78
     |                      ^
noise.javascript:1:23: error: expected digit after '.' in float literal
   1 | 123.45.67 12abc 12.34.56..78
     |                       ^~~
noise.javascript:1:26: error: illegal character '.' in program
   1 | 123.45.67 12abc 12.34.56..78
     |                          ^
noise.javascript:2:9: error: illegal character '&' in program
   2 | -+*/% &&& ||| != == <= >= >< =!= &== |= <=>
     |         ^
noise.javascript:2:13: error: illegal character '|' in program
   2 | -+*/% &&& ||| != == <= >= >< =!= &== |= <=>
     |             ^
noise.javascript:2:38: error: illegal character '|' in program
   2 | -+*/% &&& ||| != == <= >= >< =!= &== |= <=>
     |                                      ^
noise.javascript:3:53: warning: unknown escape sequence '\q'
   3 |     "unterminated string" "string with illegal char \q" ""empty""
     |                                                     ^~
noise.javascript:4:5: warning: unknown escape sequence '\e'
   4 | "bad\escape" "no closing quote
     |     ^~
noise.javascript:4:14: error: missing terminating character '"' on string literal
   4 | "bad\escape" "no closing quote
     |              ^~~~~~~~~~~~~~~~~
noise.javascript:10:21: error: unterminated comment
  10 | /* final comment */ /* another unclosed
     |                     ^~
            \end{lstlisting}
    \end{description}

\end{appendices}

%----------------------------------------------------------------------------------------

\end{document}
